\documentclass{scrartcl}

\usepackage{german}
\usepackage[utf8]{inputenc}  %Umlaute
\usepackage[T1]{fontenc}     %Umlauttrennung
\usepackage{lmodern}         %modernes Schriftbild
\usepackage{amsmath}         %math Umgebungen
\usepackage{graphicx}
\usepackage{hyperref}        %URLs
\usepackage{gensymb}         %Gradzeichen
\usepackage{float}           %Positionierung von Tabellen und Abb
\usepackage{textgreek}


\begin{document}
\begin{titlepage}
  \begin{center}
    \vspace*{1cm}
    \LARGE
    Physikpraktikum für Naturwissenschaftler \\
    \vspace*{1cm}
    \Huge
    \textbf{Versuch: Kennlinien} \\
    \vspace*{0.3cm}
    \Large
    Durchgeführt am 17. Januar 2019 \\
    Betreuer: Johannes Fendt \\
    \vspace*{2.5cm}
    Gruppe 13 \\
    Felix Burr: felix.burr@uni-ulm.de \\
    Johannes Spindler: johannes.spindler@uni-ulm.de \\
    \vfill 
  \end{center}
  Wir bestätigen hiermit, das Protokoll selbstständig erarbeitet zu haben und in genauer Kenntnis über dessen Inhalt zu sein. \\
  \vspace*{0.8cm}
  \\
  Felix Burr
  \hfill
  Johannes Spindler
\end{titlepage}
\pagebreak
\tableofcontents


\pagebreak

\section{Einleitung}
Die Stromstärke $I$ eines elektrischen Stroms ist definiert als die Ladungsmenge $\Delta Q$, die pro Zeitintervall $\Delta t$ durch einen Querschnitt des Stromkreises fließt:
\begin{align}
I = \frac{\Delta Q}{\Delta t} = \frac{dQ}{dt}
\end{align}
Das bedeutet, der in einem Material fließende Strom bei einer gegebenen Spannung hängt von der Fähigkeit des Materials ab, Ladungen zu transportieren. Diese mikroskopische Eigenschaft wird als elektrische Leitfähigkeit $\sigma$ des Materials bezeichnet. Es besteht folgender Zusammenhang mit makroskopischen Größen:
\begin{align}
\sigma = G \frac{l}{A} = \frac{I}{U} \cdot \frac{l}{A}
\end{align}
Hier bezeichnet $G = \frac{I}{U}$ den Leitwert, $l$ die Leiterlänge und $A$ die Querschnittsfläche. Diese makroskopischen Größen sind leicht messbar.

Das makroskopisch messbare elektrische Verhalten des Stromkreises wird in Form von Kennlinien dargestellt. Dazu wird eine Spannung angelegt, schrittweise variiert und jeweils der Stromfluss gemessen. Die Wertepaare werden in einem $U$-$I$-Diagramm aufgetragen. In diesem Versuch werden so die Bauteile Metallfaden- und Kohlefadenlampe, Halbleiter-Diode und MOS-FET untersucht.




\pagebreak
\section{Kennlinien von Metallfaden- und Kohlefadenlampe}
\subsection{Versuchsaufbau und Durchführung}

\begin{figure}[H]
  \centering
    \includegraphics[scale=0.75]{Aufbau1.JPG}
  \caption{Schaltbild zur Messung an einer Lampe (aus der Versuchsanleitung)}
  \label{fig:Aufbau1}
\end{figure}

Wie in Abbildung \ref{fig:Aufbau1} gezeigt, wird die Lampe an ein Netzgerät angeschlossen, mit welchem Spannungen $U$ zwischen -40V und +40V in 2,5V-Schritten angelegt werden (zwischen -5V und +5V aber 1V-Schritte). Mithilfe eines parallel-geschalteten Voltmeters kann die Spannung noch genauer eingestellt werden. Das in Reihe geschaltete Amperemeter dient zur Messung des Stroms $I$.

Anschließend werden für jede Spannung die Verlustleistung $P$ und der Widerstand $R$ berechnet:
\begin{align}
P &= U \cdot I \\
R &= \frac{U}{I}
\end{align}
Für $U = 0V$ muss der Widerstand stattdessen als Inverses der Steigung in der Kennlinie bestimmt werden:
\begin{align}
R(0V) = \frac{1V-(-1V)}{I(1V)-I(-1V)} = \frac{2V}{I(1V)-I(-1V)}
\end{align}
Damit werden die Kennlinie und das $P$-$R$-Diagramm erstellt und diskutiert.

Als Lampe wird zuerst eine Kohlefaden-, dann eine Metallfadenlampe verwendet.
\subsection{Messwerte und Ergebnisse}
Tabelle \ref{tab:Kohlefadenlampe} zeigt die Messreihe für die Kohlefadenlampe, Tabelle \ref{tab:Metallfadenlampe} für die Metallfadenlampe. Abbildung \ref{fig:V1_Kennlinien} zeigt die diesen Tabellen entsprechenden $U$-$I$-Kennlinien und Abbildung \ref{fig:V1_Leistung-Widerstand} den Widerstand $R$ aufgetragen über die Verlustleistung $P$ beider Lampen.
\begin{table}[H]
\captionof{table}{Messwerte für I und daraus errechnete Werte für P und R bei schrittweise variierter Spannung U für eine Kohlefadenlampe.}
\begin{center}
\begin{tabular}{l|l|l|l}
U [V]   &   I [mA]   &   P [W]   &   R [\textOmega]\\
\hline
-40,0   &   -24,6   &   0,984   &   1630 \\
-37,5   &   -22,9   &   0,859   &   1640 \\
-35,0   &   -21,1   &   0,739   &   1660 \\
-32,5   &   -19,5   &   0,634   &   1670 \\
-30,0   &   -17,9   &   0,537   &   1680 \\
-27,5   &   -16,2   &   0,446   &   1700 \\
-25,0   &   -14,6   &   0,365   &   1710 \\
-22,5   &   -13,1   &   0,295   &   1720 \\
-20,0   &   -11,5   &   0,230   &   1740 \\
-17,5   &   -9,9    &   0,173   &   1770 \\
-15,0   &   -8,4    &   0,126   &   1790 \\
-12,5   &   -7,0    &   0,087   &   1790 \\
-10,0   &   -5,5    &   0,055   &   1820 \\
-7,5    &   -4,0    &   0,030   &   1880 \\
-5,0    &   -2,6    &   0,013   &   1920 \\
-4,0    &   -2,1    &   0,0084  &   1900 \\
-3,0    &   -1,6    &   0,0048  &   1880 \\
-2,0    &   -1,0    &   0,0020  &   2000 \\
-1,0    &   -0,5    &   0,0005  &   2000 \\
0       &   0       &   0       &   2000 \\
+1,0    &   +0,5    &   0,0005  &   2000 \\
+2,0    &   +1,0    &   0,0020  &   2000 \\
+3,0    &   +1,6    &   0,0048  &   1880 \\
+4,0    &   +2,2    &   0,0088  &   1820 \\
+5,0    &   +2,7    &   0,014   &   1850 \\
+7,5    &   +4,1    &   0,031   &   1830 \\
+10,0   &   +5,5    &   0,055   &   1820 \\
+12,5   &   +7,0    &   0,088   &   1790 \\
+15,0   &   +8,5    &   0,128   &   1760 \\
+17,5   &   +10,0   &   0,175   &   1750 \\
+20,0   &   +11,5   &   0,230   &   1740 \\
+22,5   &   +13,0   &   0,293   &   1730 \\
+25,0   &   +14,6   &   0,365   &   1710 \\
+27,5   &   +16,2   &   0,446   &   1700 \\
+30,0   &   +17,8   &   0,534   &   1690 \\
+32,5   &   +19,5   &   0,634   &   1670 \\
+35,0   &   +21,1   &   0,739   &   1660 \\
+37,5   &   +22,8   &   0,855   &   1640 \\
+40,0   &   +24,5   &   0,980   &   1630 
\end{tabular}
\end{center}
\label{tab:Kohlefadenlampe}
\end{table}

\begin{table}[H]
\captionof{table}{Messwerte für $I$ und daraus errechnete Werte für $P$ und $R$ bei schrittweise variierter Spannung $U$ für eine Metallfadenlampe.}
\begin{center}
\begin{tabular}{l|l|l|l}
U [V]   &   I [mA]   &   P [W]   &   R [\textOmega]\\
\hline
-40,0   &   -24,5   &   0,980   &   1630 \\
-37,5   &   -23,5   &   0,881   &   1600 \\
-35,0   &   -22,5   &   0,788   &   1560 \\
-32,5   &   -21,5   &   0,699   &   1510 \\
-30,0   &   -20,5   &   0,615   &   1460 \\
-27,5   &   -19,3   &   0,531   &   1420 \\
-25,0   &   -18,3   &   0,458   &   1370 \\
-22,5   &   -17,1   &   0,385   &   1320 \\
-20,0   &   -15,9   &   0,318   &   1260 \\
-17,5   &   -14,6   &   0,256   &   1200 \\
-15,0   &   -13,3   &   0,200   &   1130 \\
-12,5   &   -11,8   &   0,148   &   1060 \\
-10,0   &   -10,3   &   0,103   &    970 \\
-7,5    &   -8,6    &   0,065   &    870 \\
-5,0    &   -6,7    &   0,034   &    750 \\
-4,0    &   -6,2    &   0,025   &    650 \\
-3,0    &   -5,3    &   0,016   &    570 \\
-2,0    &   -4,2    &   0,0084  &    480 \\
-1,0    &   -2,4    &   0,0024  &    420 \\
0       &   0       &   0       &    410 \\
+1,0    &   +2,5    &   0,0025  &    400 \\
+2,0    &   +3,2    &   0,0064  &    630 \\
+3,0    &   +4,9    &   0,015   &    610 \\
+4,0    &   +6,0    &   0,024   &    670 \\
+5,0    &   +6,6    &   0,033   &    760 \\
+7,5    &   +8,6    &   0,064   &    870 \\
+10,0   &   +10,3   &   0,103   &    970 \\
+12,5   &   +11,9   &   0,149   &   1050 \\
+15,0   &   +13,2   &   0,198   &   1140 \\
+17,5   &   +14,6   &   0,256   &   1200 \\
+20,0   &   +15,9   &   0,318   &   1260 \\
+22,5   &   +17,0   &   0,383   &   1320 \\
+25,0   &   +18,2   &   0,455   &   1370 \\
+27,5   &   +19,3   &   0,531   &   1420 \\
+30,0   &   +20,4   &   0,612   &   1470 \\
+32,5   &   +21,4   &   0,696   &   1520 \\
+35,0   &   +22,5   &   0,788   &   1560 \\
+37,5   &   +23,4   &   0,878   &   1600 \\
+40,0   &   +24,4   &   0,976   &   1640 
\end{tabular}
\end{center}
\label{tab:Metallfadenlampe}
\end{table}

\begin{figure}[H]
  \centering
    \includegraphics[scale=0.5]{V1_Kennlinien.PNG}
  \caption{Die gemessenen $U$-$I$-Kennlinien für beide Lampen}
  \label{fig:V1_Kennlinien}
\end{figure}

\begin{figure}[H]
  \centering
    \includegraphics[scale=0.5]{V1_Leistung-Widerstand.PNG}
  \caption{Widerstand $R$ über Verlustleistung $P$ aufgetragen für beide Lampen}
  \label{fig:V1_Leistung-Widerstand}
\end{figure}
\subsection{Ergebnisdiskussion}
Die Kennlinie der Kohlefadenlampe kann besser durch einen linearen Verlauf angenähert werden als die der Metallfadenlampe, welche einen stärkeren sigmoiden Charakter hat. Bei der Metallfadenlampe steigt der Strom im Bereich [-5V, +5V] am stärksten an. Im gemessenen Spannungsbereich liegt der Betrag des Stroms der Metallfadenlampe bei allen Spannungen über der Kohlefadenlampe, erst an den Randwerten nähern sich die Kurven einander an. Da der Widerstand das Inverse der Steigung der Kennlinie ist, bedeutet das, dass die Kohlefadenlampe im gemessenen Spannungsbereich einen höheren Widerstand hat, was auch im $P$-$R$-Diagramm zu sehen ist. Dort liegt der Widerstand des Kohlefadens im ganzen Bereich über dem des Metallfadens, erst am Rand (etwa 1W) sind die Widerstände nahezu identisch. Das bedeutet, dass die Metallfadenlampe wegen ihres geringeren Widerstandes bei gleicher Leistung besser geeignet ist.



\pagebreak
\section{Kennlinie einer Halbleiter-Diode}
\subsection{Versuchsaufbau und Durchführung}

\begin{figure}[H]
  \centering
    \includegraphics[scale=0.75]{Aufbau2.JPG}
  \caption{Schaltbild zur Messung an einer Halbleiter-Diode (aus der Versuchsanleitung)}
  \label{fig:Aufbau2}
\end{figure}

Der Versuchsaufbau bleibt derselbe, nur wird jetzt statt einer Lampe eine Zenerdiode eingebaut und vermessen (siehe Abbildung \ref{fig:Aufbau2}). Ziel des Versuchs ist, die Kennlinie der Diode zu erstellen. Die Spannungswerte $U$ werden so frei gewählt, dass der Verlauf der Kennlinie gut erkennbar ist, dazu wird die Stromstärke $I$ gemessen. Laut Versuchsanleitung ist eine Kennlinie wie in Abbildung \ref{fig:Kennlinie_Diode} zu erwarten. Allerdings darf der Strom nicht 200mA übersteigen, um die Diode nicht zu beschädigen.
\subsection{Messwerte und Ergebnisse}
Die Wertepaare der Kennlinie sind in Tabelle \ref{tab:Diode} aufgeführt und in Abbildung \ref{fig:V2_Kennlinie} eingetragen (bei den negativen Spannungen ist $I$ konstant null).

\begin{figure}[H]
  \centering
    \includegraphics[scale=0.5]{V2_Kennlinie.PNG}
  \caption{Die gemessene $U$-$I$-Kennlinien der Zenerdiode}
  \label{fig:V2_Kennlinie}
\end{figure}

\begin{table}[H]
\captionof{table}{Messwerte für $I$ bei variierter Spannung $U$ für eine np-Diode.}
\begin{center}
\begin{tabular}{l|l}
U [mV]   &   I [mA] \\
\hline
-2000   &     0,0 \\
-1500   &     0,0 \\
-1000   &     0,0 \\
-500    &     0,0 \\
    0   &     0,0 \\
+200    &     0,0 \\
+400    &     0,0 \\
+600    &     0,1 \\
+650    &     0,4 \\
+700    &     1,1 \\
+750    &     5,2 \\
+775    &    10,6 \\
+785    &    16,9 \\
+800    &    27,2 \\
+825    &    66,7 \\
+835    &   107,8 \\
+840    &   138,0 \\
+845    &   175,0
\end{tabular}
\end{center}
\label{tab:Diode}
\end{table}

\subsection{Ergebnisdiskussion}

\begin{figure}[H]
  \centering
    \includegraphics[scale=0.75]{Kennlinie_Diode.JPG}
  \caption{Kennlinie einer pn-Diode (aus der Versuchsanleitung)}
  \label{fig:Kennlinie_Diode}
\end{figure}

Die ermittelte Kennlinie hat einen exponentiellen Verlauf wie in Abbildung \ref{fig:Kennlinie_Diode} skizziert. Sie unterscheidet sich sehr stark von den Kennlinien im vorherigen Versuch: bis 400mV war kein Strom messbar, ab etwa 800mV erscheint die Kennlinie beinahe wie eine vertikale Gerade. Diese scheinbare Gerade wird zur $U$-Achse verlängert, um die Schleusenspannung $U_{S} = 800mV$ abzuschätzen.

Damit eignet sich die Diode, um Strom nur bei einer gewünschten Spannung, der Schleusenspannung, durchfließen zu lassen. Ab dieser Spannung steigt der Strom mit zunehmender Spannung scheinbar linear (tatsächlich exponentiell) an. 


\pagebreak
\section{Halbleiter-Diode bei Wechselspannung}
\subsection{Versuchsaufbau und Durchführung}

\begin{figure}[H]
  \centering
    \includegraphics[scale=0.75]{Aufbau3.JPG}
  \caption{Schaltbild zur Messung mit Oszillator an einer Halbleiter-Diode (aus der Versuchsanleitung)}
  \label{fig:Aufbau3}
\end{figure}

\subsection{Messwerte und Ergebnisse}

\subsection{Ergebnisdiskussion}




\pagebreak
\section{Kennlinie eines MOS-FET}
\subsection{Versuchsaufbau und Durchführung}

\begin{figure}[H]
  \centering
    \includegraphics[scale=0.75]{Aufbau4.JPG}
  \caption{Schaltbild zur Messung an einem MOS-FET (aus der Versuchsanleitung)}
  \label{fig:Aufbau4}
\end{figure}

\subsection{Messwerte und Ergebnisse}

\subsection{Ergebnisdiskussion}

\begin{figure}[H]
  \centering
    \includegraphics[scale=0.75]{Kennlinie_Gate.JPG}
  \caption{Steuerkennlinie eines selbstleitenden n-Kanal-MOS-FET (aus der Versuchsanleitung)}
  \label{fig:Kennlinie_Gate}
\end{figure}

\begin{figure}[H]
  \centering
    \includegraphics[scale=0.75]{Kennlinie_Drain.JPG}
  \caption{Arbeitskennlinie eines selbstleitenden n-Kanal-MOS-FET (aus der Versuchsanleitung)}
  \label{fig:Kennlinie_Drain}
\end{figure}

\end{document}
