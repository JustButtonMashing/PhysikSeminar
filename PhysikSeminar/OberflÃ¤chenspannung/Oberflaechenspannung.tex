\documentclass{scrartcl}

\usepackage{german}
\usepackage[utf8]{inputenc}  %Umlaute
\usepackage[T1]{fontenc}     %Umlauttrennung
\usepackage{lmodern}         %modernes Schriftbild
\usepackage{amsmath}         %math Umgebungen
\usepackage{graphicx}
\usepackage{hyperref}        %URLs
\usepackage{gensymb}         %Gradzeichen
\usepackage{float}           %Positionierung von Tabellen und Abb

\title{Physikpraktikum für Naturwissenschaftler \\ Versuch: Oberflächenspannung}
\author{Felix Burr, Johannes Spindler (Gruppe 13)}
\date{Durchgeführt am 08. November 2018}


\begin{document}
\begin{titlepage}
  \begin{center}
    \vspace*{1cm}
    \LARGE
    Physikpraktikum für Naturwissenschaftler \\
    \vspace*{1cm}
    \Huge
    \textbf{Versuch: Oberflächenspannung} \\
    \vspace*{0.3cm}
    \Large
    Durchgeführt am 08. November 2018 \\
    Betreuerin: Sabrina Hartmann \\
    \vspace*{2.5cm}
    Gruppe 13 \\
    Felix Burr: felix.burr@uni-ulm.de \\
    Johannes Spindler: johannes.spindler@uni-ulm.de \\
    \vfill 
  \end{center}
  Wir bestätigen hiermit, das Protokoll selbstständig erarbeitet zu haben und in genauer Kenntnis über dessen Inhalt zu sein. \\
  \vspace*{0.8cm}
  \\
  Felix Burr
  \hfill
  Johannes Spindler
\end{titlepage}
\pagebreak
\tableofcontents


\pagebreak

\section{Einleitung}

\section{Oberflächenspannung von Wasser, Ethanol und Kochsalzlösung (Abreißmethode)}
\subsection{Versuchsdurchführung}
\begin{align}
\sigma = \dfrac{\Delta W}{\Delta O}
\end{align}

\begin{align}
\Delta W = F \cdot \Delta h
\end{align}

\begin{align}
\Delta O = 2A = 2(2 \pi r \cdot \Delta h) = 4 \pi r \cdot \Delta h
\end{align}

\begin{align}
\sigma = \dfrac{F \cdot \Delta h}{4 \pi r \cdot \Delta h} = \dfrac{F}{4 \pi r} \label{eq:sigma}
\end{align}
\subsection{Messwerte und Ergebnisse}
\subsubsection{Messungen bei Wasser}

\begin{table}[H]
\captionof{table}{Kraft $F$ und Oberflächenspannung $\sigma$ bei Wasser}
\begin{center}
\begin{tabular}{l|l|l}
Messung    & $F$ {[}mN{]} & $\sigma$ {[}N/m{]} \\
\hline
1          & 24,0       & 0,0588    \\
2          & 24,0       & 0,0588    \\
3          & 25,5       & 0,0624    \\
Mittelwert & 24,5       & 0,0600                    
\end{tabular}
\end{center}
\label{tab:Wasser}
\end{table}
\subsubsection{Messungen bei Ethanol}

\begin{table}[H]
\captionof{table}{Kraft $F$ und Oberflächenspannung $\sigma$ bei Ethanol}
\begin{center}
\begin{tabular}{l|l|l}
Messung    & $F$ {[}mN{]} & $\sigma$ {[}N/m{]} \\
\hline
1          & 7,0       & 0,0171    \\
2          & 7,0       & 0,0171    \\
3          & 7,0       & 0,0171    \\
Mittelwert & 7,0       & 0,0171                    
\end{tabular}
\end{center}
\label{tab:Ethanol}
\end{table}
\subsubsection{Messungen bei Kochsalzlösung}

\begin{table}[H]
\captionof{table}{Kraft $F$ und Oberflächenspannung $\sigma$ bei Kochsalzlösung}
\begin{center}
\begin{tabular}{l|l|l}
Messung    & $F$ {[}mN{]} & $\sigma$ {[}N/m{]} \\
\hline
1          & 22,0       & 0,0539    \\
2          & 20,0       & 0,0490    \\
3          & 19,0       & 0,0465    \\
Mittelwert & 20,3       & 0,0498                    
\end{tabular}
\end{center}
\label{tab:Kochsalz}
\end{table}
\subsection{Fehlerrechnung}
Aus Gleichung \ref{eq:sigma} folgt für $\sigma$, das von den Messgrößen $F$ und $r$ abhängt, der Größtfehler
\begin{align*}
\Delta \sigma = \left| \dfrac{\partial \sigma}{\partial F} \right| \Delta F + \left| \dfrac{\partial \sigma}{\partial r} \right| \Delta r = \left| \dfrac{1}{4 \pi r} \right| \Delta F + \dfrac{F}{4 \pi} \left| - \dfrac{1}{r^2} \right| \Delta r =\dfrac{1}{4 \pi r} \Delta F + \dfrac{F}{4 \pi r^2} \Delta r
\end{align*}
Da die Skala auf dem Kraftmesser in Abständen von einem Milli-Newton aufgetragen ist und zum Messen des Ringradius eine Schieblehre verwendet wurde, werden $\Delta F = 1mN, \Delta r = 0,05mm$ als Größtfehler angenommen.

\subsubsection{Standardabweichung und Größtfehler von $\sigma$ bei Wasser}
Aus Tabelle \ref{tab:Wasser} folgt eine Standardabweichung von 0,0021. Mit den Mittelwerten $F = 24,5mN, \sigma = 0,0600N/m$ ergibt sich der Größtfehler von $\sigma$:
\begin{align*}
\Delta = \sigma \dfrac{1}{4 \pi \cdot 0,0325m} 0,0010N + \dfrac{0,0245N}{4 \pi (0,0325m)^2} 5 \cdot 10^{-5}m = 0,00254N
\end{align*}
\subsubsection{Standardabweichung und Größtfehler von $\sigma$ bei Ethanol}
Wie in Tabelle \ref{tab:Ethanol} zu sehen ist, wurde in allen drei Messungen derselbe Wert für die Kraft gemessen, weshalb die Standardabweichung Null beträgt. Mit den Mittelwerten $F = 7,0mN, \sigma = 0,0171N/m$ ergibt sich der Größtfehler von $\sigma$:
\begin{align*}
\Delta = \sigma \dfrac{1}{4 \pi \cdot 0,0325m} 0,0010N + \dfrac{0,0070N}{4 \pi (0,0325m)^2} 5 \cdot 10^{-5}m = 0,00247N
\end{align*}
\subsubsection{Standardabweichung und Größtfehler von $\sigma$ bei Kochsalzlösung}
Aus Tabelle \ref{tab:Kochsalz} folgt eine Standardabweichung von 0,0037. Mit den Mittelwerten $F = 20,3mN, \sigma = 0,0498N/m$ ergibt sich der Größtfehler von $\sigma$:
\begin{align*}
\Delta = \sigma \dfrac{1}{4 \pi \cdot 0,0325m} 0,0010N + \dfrac{0,0203N}{4 \pi (0,0325m)^2} 5 \cdot 10^{-5}m = 0,00253N
\end{align*}
\subsection{Ergebnisdiskussion}


\section{Oberflächenspannung von Tensidlösungen}
\subsection{Versuchsdurchführung}
\subsection{Messwerte und Ergebnisse}
\subsection{Ergebnisdiskussion}


\section{Oberflächenspannung von Wasser und einer SDS-Lösung (Kapillarmethode)}
\subsection{Versuchsdurchführung}
\subsection{Messwerte und Ergebnisse}
\subsection{Ergebnisdiskussion}


\section{Benetzung von Oberflächen}
\subsection{Versuchsdurchführung}
\subsection{Messwerte und Ergebnisse}
\subsection{Ergebnisdiskussion}


\end{document}